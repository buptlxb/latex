\documentclass[a4paper, 11pt]{article}
\title{Homework 2}
\author{ICT-LXB}
\begin{document}
\maketitle
\tableofcontents
\section{There Loops Resolution}

\subsection{Source Code}
\begin{verbatim}
#include <stdio.h>
#include <stdlib.h>
#include <time.h>
#include <string.h>
#include <sys/time.h>

#define DIM 4096
typedef double(* MATRIX)[DIM];

void multiply(MATRIX z, const MATRIX x, const MATRIX y, int dim)
{
    int row, col, idx, tmp;
    for (row = 0; row < dim; row++)
        for (col = 0; col < dim; col++) {
            tmp = 0;
            for (idx = 0; idx < dim; idx++)
                tmp += x[row][idx] * y[idx][col];
            z[row][col] = tmp;
        }
}

void init_matrix(MATRIX m, int dim, int isRandom)
{
    int row, col;
    if (isRandom) {
        srandom(time(NULL));
        for (row = 0; row < dim; row++)
            for (col = 0; col < dim; col++)
                m[row][col] = random() % 263;
    } else
        memset((void *)m, 0, dim * dim * sizeof(m[0][0]));
}

void print_matrix(MATRIX m, int dim)
{
    int row, col;
    for (row = 0; row < dim; row++) {
        for (col = 0; col < dim; col++)
            printf("%f ", m[row][col]);
        printf("\n");
    }
}

int main(void)
{
    MATRIX x = (MATRIX)malloc(DIM*DIM*sizeof(double));
    MATRIX y = (MATRIX)malloc(DIM*DIM*sizeof(double));
    MATRIX z = (MATRIX)malloc(DIM*DIM*sizeof(double));
    init_matrix(x, DIM, 1);
    init_matrix(y, DIM, 1);
    init_matrix(z, DIM, 0);
    struct timeval start, end;
    gettimeofday(&start, NULL);
    multiply(z, x, y, DIM);
    gettimeofday(&end, NULL);
    int time = (int)(end.tv_sec-start.tv_sec);
    unsigned long long opts = (unsigned long long)DIM*DIM*(2*DIM-1);
    double speed = opts / (double)(time * 1000000);
    printf("%d matrix multiply\n", DIM);
    printf("time : %ds\tspeed: %.2f MFLOPS\n", time, speed);
    return 0;
}
\end{verbatim}

\subsection{Results}

\subsubsection{No Optimization}
\begin{verbatim}
    ict-lxb@ictlxb-Zhaoyang-E49:~/Workspace/test$ ./matrix
    4096 matrix multiply
    time : 1065s    speed: 129.03 MFLOPS
\end{verbatim}

\subsubsection{-O2 Optimization}
\begin{verbatim}
    ict-lxb@ictlxb-Zhaoyang-E49:~/Workspace/test$ ./matrix
    4096 matrix multiply
    time : 813s speed: 169.03 MFLOPS
\end{verbatim}

\subsection{Conclusion}
After analysing the results above, we can see that the same source code
compiled with different optimization options might result in very different
performance. The compiler can improve the performance by reordering the
instructions, which might cause bubble in the instruction parallel, and so
on.

\section{Three Loops Resulotion (4097X4097)}

\subsection{Source Code}
\begin{verbatim}
#include <stdio.h>
#include <stdlib.h>
#include <time.h>
#include <string.h>
#include <sys/time.h>

#define DIM 4097
typedef double(* MATRIX)[DIM];

void multiply(MATRIX z, const MATRIX x, const MATRIX y, int dim)
{
    int row, col, idx, tmp;
    for (row = 0; row < dim; row++)
        for (col = 0; col < dim; col++) {
            tmp = 0;
            for (idx = 0; idx < dim; idx++)
                tmp += x[row][idx] * y[idx][col];
            z[row][col] = tmp;
        }
}

void init_matrix(MATRIX m, int dim, int isRandom)
{
    int row, col;
    if (isRandom) {
        srandom(time(NULL));
        for (row = 0; row < dim; row++)
            for (col = 0; col < dim; col++)
                m[row][col] = random() % 263;
    } else
        memset((void *)m, 0, dim * dim * sizeof(m[0][0]));
}

void print_matrix(MATRIX m, int dim)
{
    int row, col;
    for (row = 0; row < dim; row++) {
        for (col = 0; col < dim; col++)
            printf("%f ", m[row][col]);
        printf("\n");
    }
}

int main(void)
{
    MATRIX x = (MATRIX)malloc(DIM*DIM*sizeof(double));
    MATRIX y = (MATRIX)malloc(DIM*DIM*sizeof(double));
    MATRIX z = (MATRIX)malloc(DIM*DIM*sizeof(double));
    init_matrix(x, DIM, 1);
    init_matrix(y, DIM, 1);
    init_matrix(z, DIM, 0);
    struct timeval start, end;
    gettimeofday(&start, NULL);
    multiply(z, x, y, DIM);
    gettimeofday(&end, NULL);
    int time = (int)(end.tv_sec-start.tv_sec);
    unsigned long long opts = (unsigned long long)DIM*DIM*(2*DIM-1);
    double speed = opts / (double)(time * 1000000);
    printf("%d matrix multiply\n", DIM);
    printf("time : %ds\tspeed: %.2f MFLOPS\n", time, speed);
    return 0;
}
\end{verbatim}

\subsection{Results}

\begin{enumerate}
    \item
\begin{verbatim}
    ict-lxb@ictlxb-Zhaoyang-E49:~/Workspace/test$ ./matrix
    4097 matrix multiply
    time : 801s speed: 171.69 MFLOPS
\end{verbatim}
    \item
\begin{verbatim}
    ict-lxb@ictlxb-Zhaoyang-E49:~/Workspace/test$ ./matrix
    4097 matrix multiply
    time : 880s speed: 156.28 MFLOPS
\end{verbatim}
    \item
\begin{verbatim}
    ict-lxb@ictlxb-Zhaoyang-E49:~/Workspace/test$ ./matrix
    4097 matrix multiply
    time : 795s speed: 172.98 MFLOPS
\end{verbatim}
\end{enumerate}

\emph{Average Time: 825s; Average Speed: 166.98 MFLOPS}

\subsection{Conclusion}
After analysing the results above, we can see that the performance migth vary
from time to time. Comparing the average time with the time of Experiment One,
we can see that the time does not increase too much and the FLOPS decreases a
little. The reasom might be the cache miss rates increase a bit.


\section{Three Loops Resulotion With Transpose}

\subsection{Source Code}
\begin{verbatim}
#include <stdio.h>
#include <stdlib.h>
#include <time.h>
#include <string.h>
#include <sys/time.h>

#define DIM 4097
typedef double(* MATRIX)[DIM];

void transpose(MATRIX m, int dim)
{
    int row, col, tmp;
    for (row = 0; row < dim; row++)
        for (col = 0; col < dim; col++) {
            if (col == row)
                continue;
            tmp = m[row][col];
            m[row][col] = m[col][row];
            m[col][row] = tmp;
        }
}

void multiply(MATRIX z, const MATRIX x, const MATRIX y, int dim)
{

    int row, col, idx, tmp;
    transpose(y, dim);
    for (row = 0; row < dim; row++)
        for (col = 0; col < dim; col++) {
            tmp = 0;
            for (idx = 0; idx < dim; idx++)
                tmp += x[row][idx] * y[col][idx];
            z[row][col] = tmp;
        }
}

void init_matrix(MATRIX m, int dim, int isRandom)
{
    int row, col;
    if (isRandom) {
        srandom(time(NULL));
        for (row = 0; row < dim; row++)
            for (col = 0; col < dim; col++)
                m[row][col] = random() % 263;
    } else
        memset((void *)m, 0, dim * dim * sizeof(m[0][0]));
}

void print_matrix(MATRIX m, int dim)
{
    int row, col;
    for (row = 0; row < dim; row++) {
        for (col = 0; col < dim; col++)
            printf("%f ", m[row][col]);
        printf("\n");
    }
}

int main(void)
{
    MATRIX x = (MATRIX)malloc(DIM*DIM*sizeof(double));
    MATRIX y = (MATRIX)malloc(DIM*DIM*sizeof(double));
    MATRIX z = (MATRIX)malloc(DIM*DIM*sizeof(double));
    init_matrix(x, DIM, 1);
    init_matrix(y, DIM, 1);
    init_matrix(z, DIM, 0);
    struct timeval start, end;
    gettimeofday(&start, NULL);
    multiply(z, x, y, DIM);
    gettimeofday(&end, NULL);
    int time = (int)(end.tv_sec-start.tv_sec);
    unsigned long long opts = (unsigned long long)DIM*DIM*(2*DIM-1);
    double speed = opts / (double)(time * 1000000);
    printf("%d matrix multiply\n", DIM);
    printf("time : %ds\tspeed: %.2f MFLOPS\n", time, speed);
    return 0;
}
\end{verbatim}

\subsection{Results}
\begin{enumerate}
    \item
\begin{verbatim}
    ict-lxb@ictlxb-Zhaoyang-E49:~/Workspace/test$ ./matrix_transpose 
    4097 matrix multiply
    time : 234s speed: 587.70 MFLOPS
\end{verbatim}
    \item
\begin{verbatim}
    ict-lxb@ictlxb-Zhaoyang-E49:~/Workspace/test$ ./matrix_transpose 
    4097 matrix multiply
    time : 235s speed: 585.20 MFLOPS
\end{verbatim}
    \item
\begin{verbatim}
    ict-lxb@ictlxb-Zhaoyang-E49:~/Workspace/test$ ./matrix_transpose 
    4097 matrix multiply
    time : 234s speed: 587.70 MFLOPS
\end{verbatim}
\end{enumerate}

\emph{Average Time: 234.33s; Average Speed: 586.87 MFLOPS}

\subsection{Conclusion}
After analysing the results above, we can see that the performance is improved
so much. The reasons for that might be that the array in C program language is
row-major and that cache miss rate decreases due to the transpose. Temporal
locality and spatial locality is the key point in this improvement.

\section{Three Loops Resulotion With Partition}

\subsection{Source Code}
\begin{verbatim}
#include <stdio.h>
#include <stdlib.h>
#include <time.h>
#include <string.h>
#include <sys/time.h>

#define B 512
#define DIM 4096
typedef double(* MATRIX)[DIM];

#define MIN(x, y) ((x) > (y) ? (y) : (x))

void multiply(MATRIX z, const MATRIX x, const MATRIX y, int dim)
{
    int row, col, idx, ri, ci, tmp;
    for (row = 0; row < dim; row+=B)
        for (col = 0; col < dim; col+=B)
            for (ri = 0; ri < dim; ri++)
                for (ci = col; ci < MIN(dim, col+B); ci++) {
                    tmp = 0;
                    for (idx = col; idx < MIN(dim, col+B); idx++)
                        tmp += x[ri][idx] * y[idx][ci];
                    z[ri][ci] += tmp;
                }
}

void init_matrix(MATRIX m, int dim, int isRandom)
{
    int row, col;
    if (isRandom) {
        for (row = 0; row < dim; row++)
            for (col = 0; col < dim; col++)
                m[row][col] = random() % 263;
    } else
        memset((void *)m, 0, dim * dim * sizeof(m[0][0]));
}

void print_matrix(MATRIX m, int dim)
{
    int row, col;
    for (row = 0; row < dim; row++) {
        for (col = 0; col < dim; col++)
            printf("%f ", m[row][col]);
        printf("\n");
    }
}

int main(void)
{
    srandom(time(NULL));
#if 1
    MATRIX x = (MATRIX)malloc(DIM*DIM*sizeof(double));
    MATRIX y = (MATRIX)malloc(DIM*DIM*sizeof(double));
    MATRIX z = (MATRIX)malloc(DIM*DIM*sizeof(double));
    init_matrix(x, DIM, 1);
    init_matrix(y, DIM, 1);
    init_matrix(z, DIM, 0);
#else
    double x[DIM][DIM] = {
        {1, 2},
        {3, 4}
    };
    double y[DIM][DIM] = {
        {4, 3},
        {2, 1}
    };
    double z[DIM][DIM] = {
        {0, },
    };
#endif
    struct timeval start, end;
    gettimeofday(&start, NULL);
    multiply(z, x, y, DIM);
    gettimeofday(&end, NULL);
    int time = (int)(end.tv_sec-start.tv_sec);
    unsigned long long opts = (unsigned long long)DIM*DIM*(2*DIM-1);
    double speed = opts / (double)(time * 1000000);
    printf("%d matrix multiply\n", DIM);
    printf("time : %ds\tspeed: %.2f MFLOPS\n", time, speed);
#if 0
    print_matrix(x, DIM);
    print_matrix(y, DIM);
    print_matrix(z, DIM);
#endif
    return 0;
}
\end{verbatim}

\subsection{Results}
picture

\subsection{Conclusion}
After analysing the results above, we can see that the performance is optimal
when B is 64.

\section{Serial Program Analysis}

\section{Optimal Program With Pthread}

\end{document}
